\section{Limitations of the Study}
\label{sec:limitations}

Although this study follows a systematic methodology for selecting the included works in the review, the methodology followed by this study has limitations. One limitation is related to the goal of overview the long-term SLAM topic instead of providing an in-depth analysis, discussion, and comparison focused on a single challenge intrinsic to lifelong autonomy. This limitation leads to an extensive and complex discussion of the included works that may possibly not cover all the details of the proposed methods in the included records.
However, none of the existent reviews in SLAM focus on the long-term localization and mapping problem nor clarify the selection methodology of the works included in their studies. Also, not focusing on a single challenge related to long-term SLAM provides a broader and interesting discussion of the topic, given that some challenges may be related between each other. For example, removing elements from the current map estimation considered to be outdated due to appearance changes is related to both environment changing conditions and map management.
Even so, the broader discussion itself is a limitation of the study, and future ones may prefer to focus only on methodologies related to a single challenge of lifelong SLAM.

Furthermore, the quality assessment in the selection phase of this study only considers 2 criteria associated with the topic of this study, namely, QE3 and QE5. While the remaining quality criteria evaluate the eligible records in terms of their scientific contribution, only 2 out of 9 being related to the topic of the review may be considered a limitation of this study. This limitation is related to the previous one. Indeed, the QE not considering specific challenges and characteristics of long-term autonomy tries to avoid the a priori knowledge of the authors to the review on lifelong SLAM biasing the methods' selection methodology.
However, even though the followed methodology obtained two distinct peaks in the QE scores (see Figure~\ref{fig:methodology:qe}) that could be interpreted as belonging to 2 different clusters -- records to exclude versus the ones to include in the review --, the inclusion of more QE criteria would perhaps improve the distinction between the clusters.

The other limitation of this study is the discussion of the public datasets. Instead of considering a different query to find and select datasets, the discussion focused only on the ones used in the experiments by the included works. While this selection approach allowed the identification of 43 different datasets, it does not mean that these datasets are the best to use to evaluate methodologies related to long-term SLAM due to the identification of the datasets discussed in this study may be biased by the included works themselves. Although the main focus of this systematic literature review is on ways to achieve lifelong autonomy and not only on the experimental data, future studies should consider to review separately the datasets from the methodologies. Still, the inclusion of an analysis of common used datasets by the included works improves the interest of this review for researchers interested in long-term localization and mapping.
