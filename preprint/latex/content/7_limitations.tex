\section{Limitations of the Study}
\label{sec:limitations}

Section to discuss possible limitations of the study (timeframe, wider approach, etc.).

\begin{itemize}[nosep]
\item only one query for discussion: e.g., for searching datasets, possibly, a different query should have been used
\item overview long-term SLAM vs in-depth analysis and discussion of each type of techniques: our review synthesizes all types of techniques, if the reader wants an in-depth analysis, different reviews should be performed
\item related to previous one, each category appearance dynamics sparsity multi-session and computational have all different aspects related to each other, that probably should be treated differently in different data extraction items to improve the data organization -- however the main goal of this review was to overview all trends and not focus on one specifically, so this categorization should be done separately and probably in singular literature reviews.
\item limited information on the experiments conditions (traveled distance, duration, etc.) given by the authors
\item some works such as PointNetVLAD, FAB-MAP and SeqSLAM (+LOAM, and LEGO-LOAM) did not appeared in the identification phase. However, the ones included have improvements over all of these...; also ``Robust optimization of factor graphs by using condensed measurements''
\item discussion does not focus on the datasets used but rather on the methodologies, to not extend even further the review
\item ORBSLAM2, ORBSLAM3 not identified
\item in retrospective, a bias of the authors may occurred relative and influenced in the inclusion of articles related to air-vehicles, underwater, and multi-robot systems -- air although batteries possibly a limitation their viewpoint specially on bird eye camera view is completely different than usual ground robots (facing forward) and so long-term methods for ground may not be as effective due to different viewpoint, underwater due to low number of articles found and the importance of long-term SLAM in these environments (plants growing, different conditions of the water in camera), and multi-robot systems due to the future these systems can improve greatly the mapping and planning efficiency of robotic installations
\end{itemize}
