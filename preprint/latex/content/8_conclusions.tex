\section{Conclusions}
\label{sec:conclusions}

This paper presents a systematic literature review on long-term localization and mapping for mobile robots. The review selects 142 works from the literature covering the main strategies to achieve lifelong SLAM and discussing the experimental data (including private experiments and public datasets) and evaluation metrics commonly used to assess the performance of autonomous systems in long-term operations.
The discussion analyzes the included works in terms of appearance invariance to changing conditions in the environment, dealing with dynamic elements in the scene, multi-session strategies, map sparsification techniques to bound the computational resources, and other computational-related topics.
Also, an overview over the bibliographic data of the included works identifies the most used terms in long-term SLAM and identifying research networks between authors using the VOSviewer~\parencite{results:vosviewer:1,results:vosviewer:2} tool, while also discussing the evolution of the number of publications over time and the publication venues with more records.

Overall, the methodologies discussed in this study are a step forward to achieve lifelong autonomy.
In terms of dealing with appearance changes in the environment, CNN-based features are more discriminative compared to handcrafted features, while considering both appearance and geometric cues in the descriptor improve its invariance to changing conditions. Semantic features and considering different sensorization sources (e.g., cameras, LiDAR, and/or radar) can also increase the robustness to appearance changes in the environment.
For dealing with dynamic elements, the most common approach is to distinguish between static, dynamic, and semi-static changes in the perceived environment, while only using static permanent changes for localization to improve the reliability of the pose estimator.
As for constraining the map size to the explored environment area instead of the trajectory length, the evaluation of the mutual information by using techniques based on information theory or heuristics is the most currently used technique to bound the map size.

However, there are still challenges and future directions for the research on long-term SLAM. Vision-based global recognition is affected by viewpoint variance, where using omnidirectional sensors to increase the field-of-view relative to perspective cameras may decrease that variance. Decentralized computation architectures may be an interesting research direction to offload heavier computation tasks from the mapping agent and improve the scalability of long-term SLAM algorithms. For example, cloud and edge computing devices allow the parallelization of tasks and increase the availability of computational resources, even though communication constraints should be accounted. Also, the extension of the methodologies discussed in this study to multi-robots systems may optimize the operation of autonomous systems while also increasing their robustness in the long-term.

Lastly, this review can be updated in future studies thanks to following a systematic review that defines explicitly the time interval in the selection process (the data of the last full inquiry is May 17, 2022) and the search query used to identify records from the literature.
Also, all documentation and scripts using during the review process are available in a public GitHub repository to facilitate replicating of the results and future updates to this review.
This paper may be extended to focus on single challenges of long-term SLAM for providing an in-depth comparison with experimental results between different methodologies.
