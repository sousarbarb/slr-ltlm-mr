\section{Purpose of the study}
\label{sec:purpose}

\textcolor{red}{PUT INTRO HERE!!!}





\subsection{Limitations of current studies}



\begin{table}[h]
  \caption[Existent Literature Reviews, Surveys, and Tutorials on SLAM.]{Existent Literature Reviews, Surveys, and Tutorials on SLAM.}
  \label{tab:purpose:current-literature}
  \centering
  {\scriptsize
  \begin{tabular}{c p{0.43\columnwidth} p{0.43\columnwidth}}

\hline
\textbf{Year} & \textbf{Topic} & \textbf{Reference}\\
\hline
2006 & Probabilistic formulations (EKF, particle filter)
     & \cite{purpose:study:durrant-whyte-bailey:2006:1}, \cite{purpose:study:durrant-whyte-bailey:2006:2}\\
\hline
2008 & Probabilistic and pose graph formulations
     & \cite{purpose:study:thrun:2008}\\
\hline
2010 & GraphSLAM
     & \cite{purpose:study:grisetti:2010}\\
\hline
2011 & Observability, convergence, consistency (feature-based SLAM)
     & \cite{purpose:study:dissanayake:2011}\\
\hline
2012 & Visual odometry
     & \cite{purpose:study:scaramuzza-fraundorfer:2011:1}, \cite{purpose:study:scaramuzza-fraundorfer:2012:2}\\
\hline
2014 & Underwater navigation and localization
     & \cite{purpose:study:paull:2014}\\
\hline
2015 & Visual SLAM
     & \cite{purpose:study:yousif:2015}\\
\hline
2016 & Observability, convergence, consistency (feature and graph-based SLAM)
     & \cite{purpose:study:huang-dissanayake:2016}\\
     & Multi-robot SLAM
     & \cite{purpose:study:saeedi:2016}\\
     & Visual place recognition
     & \cite{purpose:study:lowry:2016}\\
     & SLAM literature overview
     & \cite{purpose:study:cadena:2016}\\
\hline
2017 & Autonomous vehicles
     & \cite{purpose:study:bresson:2017}\\
\cline{1-1}
\cline{3-3}
2018 &
     & \cite{purpose:study:kuutti:2018}\\
\hline
2018 & AI for long-term autonomy
     & \cite{purpose:study:kunze:2018}\\
     & Long-term sensorization
     & \cite{purpose:study:zaffar:2018}\\
\hline
2020 & Deep learning
     & \cite{purpose:study:fayyad:2020}\\
     & Multi-robot search and rescue
     & \cite{purpose:study:queralta:2020}\\
\hline
2021 & Self-driving vehicles
     & \cite{purpose:study:badue:2021}\\
\hline
2022 & Underground navigation
     & \cite{purpose:study:ebadi}\\
\hline

  \end{tabular}}
\end{table}





\subsection{Motivations and goals}

Research question: What is the current state of the art of long-term localization and mapping using mobile robots?

Goals of this review:

\begin{itemize}[nosep]
\item which are the main strategies for accomplishing long-term operations with mobile robots;
\item how to deal with varying conditions of the environment;
\item how do autonomous robots deal with the dynamics of the environment;
\item which are the main strategies to deal with the limited computational resources of a mobile robot on long-term operations;
\item how the methods evaluate their results;
\item which are the public datasets more used for evaluating long-term localization and mapping.
\end{itemize}

PICO framework (Population--Intervention--Comparison--Outcome) helps to frame the research questions of this systematic review into a more structured framework:

\begin{itemize}[nosep]
\item \textbf{Population:} mobile robots;
\item \textbf{Intervention:} localization, mapping, SLAM;
\item \textbf{Comparison:} \textit{not applicable to this study};
\item \textbf{Outcome:} long-term operation, lifelong autonomy, robust.
%\item \textbf{Context:} continuous operation, service robots, industrial environments.
\end{itemize}

